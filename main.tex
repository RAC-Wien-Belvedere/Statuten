\documentclass{statutclass}

\title{Statuten}
\author{Wien-Belvedere}
\date{0000-00-00}  % Placeholder for date, replaced by release workflow

\begin{document}
\maketitle
\tableofcontents
\clearpage

\section{Name und Sitz}
\begin{enumerate}
    \item Der Name des Vereins ist ROTARACT CLUB WIEN-BELVEDERE. Der abgekürzte Name des Vereins lautet RAC WIEN-BELVEDERE.
    \item Sitz des Vereins ist Wien.
\end{enumerate}

\section{Absicht und Zielsetzung}
\begin{enumerate}
    \item Die Tätigkeit des Vereins ist nicht auf Gewinn ausgerichtet.
    \item Der Verein ist überparteilich und an keine Konfession gebunden.
    \item Ziele des Rotaract Clubs Wien-Belvedere sind:
    \begin{enumerate}
        \item seinen Mitgliedern zu helfen, Führungseigenschaften, insbesondere die Fähigkeit zu konstruktiver Kritik, und persönliche Integrität in der Gemeinschaft zu entwickeln;
        \item den Respekt vor den Mitmenschen zu fördern, für ethisches Verhalten im Beruf einzutreten und die Würde und den Wert aller Berufe zu propagieren;
        \item die Bereitschaft der Mitglieder zum Engagement und zur Zusammenarbeit für soziale Projekte sowie ihr Verantwortungsbewusstsein zu fördern;
        \item Möglichkeiten für junge Menschen anzubieten, sich Notständen und Belangen im Gemeinwesen in Österreich und weltweit zuwenden zu können;
        \item Möglichkeiten zur Zusammenarbeit mit Rotary-Patenclubs anzubieten;
        \item jungen Menschen Motivationsanreize für eine spätere Mitgliedschaft in Rotary zu bieten.
    \end{enumerate}
    \item Die Ziele des Vereins sollen insbesondere verwirklicht werden durch:
    \begin{enumerate}
        \item ständigen gegenseitigen Gedankenaustausch bei Diskussionen und Referaten über aktuelle Themen;
        \item die Anregung und selbständige Durchführung von Aktivitäten im Dienste der Gemeinschaft;
        \item Förderung der kulturellen Interessen und Anregungen zur sinnvollen Freizeitgestaltung der einzelnen Mitglieder;
    \item Aufnahme und ständige Pflege internationaler Kontakte.
    \end{enumerate}
\end{enumerate}

\section{Finanzielle Mittel}
Die finanziellen Mittel des Vereins werden aufgebracht durch:
\begin{enumerate}[label=\alph*)]
    \item Mitgliedsbeiträge
    \item Spenden
    \item Einnahmen aus Veranstaltungen des Clubs
    \item Förderungen von Rotary Clubs und Rotary International
    \item Förderungen privater Personen und öffentlicher Stellen
    \item Zuwendungen von Sponsoren.
\end{enumerate}

\section{Beziehungen zu Rotary}
\begin{enumerate}
    \item Patenclub des Rotaract Clubs Wien-Belvedere ist der Rotary Club Wien-West, der dem Rotaract Club beratend und unterstützend zur Seite steht.
    \item Die Tätigkeit des Rotaract Clubs Wien-Belvedere soll stets harmonisch mit der Politik von Rotary International übereinstimmen. Die Beziehungen bestehen in gegenseitiger Freundschaft, Unterstützung und Hilfsbereitschaft.
    \item Aus einer bestehenden oder vergangenen Mitgliedschaft beim Rotaract Club Wien- Belvedere entsteht kein Anspruch auf eine Mitgliedschaft bei einem Rotary Club.
\end{enumerate}

\section{Arten der Mitgliedschaft}
\begin{enumerate}
    \item Es gibt ordentliche Mitglieder und außerordentliche Mitglieder. Zu den außerordentlichen Mitgliedern zählen beurlaubte Mitglieder, Ehrenmitglieder und unterstützende Mitglieder.
    \item Als \textbf{ordentliche Mitglieder} werden unbescholtene Personen mit Führungsqualitäten jeglicher Nationalität ab dem vollendeten 18. Lebensjahr aufgenommen, die in Wien oder in der Umgebung von Wien arbeiten, wohnen oder studieren, und die in keinem anderen Rotaract Club als ordentliches oder beurlaubtes Mitglied geführt werden.
    \item Der Vorstand kann jedes ordentliche Mitglied, das für eine bestimmte Zeit außerstande ist, an den Zusammenkünften des Clubs teilzunehmen, auf dessen begründeten Antrag \textbf{beurlauben}. Die Dauer der Beurlaubung ist im Vorhinein zu bestimmen, kann jedoch vom Vorstand verlängert werden. Sie kann insgesamt bis zu 12 Monate betragen, im Falle eines dauernden Aufenthalts des Mitglieds außerhalb von Wien und der Umgebung von Wien bis zu 36 Monate. Nach Ausschöpfung der 12 beziehungsweise 36 Monate ist eine neuerliche Beurlaubung frühestens nach weiteren 12 Monaten möglich.
    \item Jede natürliche Person, die kein ordentliches oder beurlaubtes Mitglied ist, kann durch Beschluss der Vollversammlung auf Antrag des Vorstandes \textbf{Ehrenmitglied} werden. Die Ehrenmitgliedschaft kann an ehemalige Clubmitglieder sowie andere Personen verliehen werden, die sich durch außergewöhnliches Engagement um den Verein und Rotaract International besonders verdient gemacht haben. Die Dauer der Ehrenmitgliedschaft ist zeitlich nicht begrenzt.
    \item Jedes ordentliche Mitglied, sowie jedes freiwillig ausgetretene oder aufgrund des Erreichens der Altersgrenze ausgeschiedene ehemals ordentliche Mitglied hat die Möglichkeit über Antrag an den Vorstand um eine \textbf{unterstützende Mitgliedschaft} anzusuchen. Der Vorstand entscheidet über diesen Antrag bei einer Vorstandssitzung. Im positiven Falle wird die ansuchende Person zu einem unterstützenden Mitglied und eine etwaige aufrechte ordentliche Mitgliedschaft erlischt.
\end{enumerate}

\section{Aufnahme von ordentlichen Mitglieder}
\begin{enumerate}
    \item Jeder Gast soll durch ein Mitglied in den Club eingeführt und betreut werden. Dieses Mitglied gilt als Pate. Die Aufgabe des Paten ist es, den Gast vorzustellen und als Kontaktperson zwischen dem Club und dem Gast während des Aufnahmeverfahrens zu dienen. Er hat den Gast über das laufende Programm sowie die anstehenden Sozialprojekte zu informieren. Weiteres hat er dem Vorstand zu melden, wenn der Gast als Mitglied aufgenommen zu werden wünscht.
    \item Wird ein Gast nicht durch ein Mitglied in den Club eingeführt und zeigt er Interesse am Besuch künftiger Veranstaltungen, ist es die Aufgabe des Vizepräsidenten, dem Gast bei der Beistellung eines Paten behilflich zu sein.
    \item Hat ein Gast, der dem Club beitreten möchte, reges Interesse am Clubleben gezeigt, insbesondere durch regelmäßiges Erscheinen bei Veranstaltungen des Clubs, und an mindestens einer Aktivität im Sozialbereich des Clubs teilgenommen, kann er auf Antrag des Paten oder eines anderen Mitglieds als ordentliches Mitglied aufgenommen werden. Wenn der Vorstand die Aufnahme des Gastes ohne Gegenstimme befürwortet, erfolgt in einer Aussendung an die ordentlichen und beurlaubten Mitglieder der Antrag zur Aufnahme im Regelfall durch den Paten, ansonsten durch den Präsidenten. Der Antragsteller hat den Kandidaten nach den Gründen für seinen Beitrittswunsch zu befragen. Der Aufnahmeantrag hat eine Beschreibung des Kandidaten sowie die Gründe für seinen Beitrittswunsch zu enthalten.
    \item Jedes ordentliche sowie jedes beurlaubte Mitglied kann gegen die Aufnahme des ausgeschriebenen Kandidaten binnen vierzehn Tagen schriftlich einen begründeten Einspruch an den Vorstand richten.
    \item Werden fristgerecht ein oder mehrere begründete Einsprüche an den Vorstand gerichtet und können diese durch eine Aussprache zwischen den Mitgliedern, welche Einspruch erhoben haben, dem Vorstand und dem Paten nicht beseitigt werden, wird die Aufnahme des Kandidaten
    \begin{enumerate}
        \item sofort abgelehnt, wenn von einem Drittel oder mehr der ordentlichen Mitglieder fristgerecht Einspruch erhoben wurde;
        \item\label{abgelehnt} sofort abgelehnt, wenn zwei Drittel des Vorstandes fristgerecht Einspruch erhoben haben.
        \item auf eine vom Vorstand aus diesem Grund einzuberufende Vollversammlung vertagt, wenn weniger als ein Drittel aller ordentlichen Mitglieder fristgerecht Einspruch erhoben haben. In dieser Vollversammlung hat eine Diskussion über die Einsprüche gegen die Aufnahme des Kandidaten zu erfolgen. Daran hat sich eine geheime Abstimmung anzuschließen. Werden drei oder mehr Gegenstimmen abgegeben, ist die Aufnahme des Kandidaten abgelehnt.
    \end{enumerate}
    \item Der Kandidat gilt als aufgenommen, wenn keine Einsprüche fristgerecht an den Vorstand gerichtet wurden oder wenn der Antrag auf Aufnahme des Kandidaten nach \ref{abgelehnt} die erforderliche Mehrheit in der Vollversammlung erhalten hat.
\end{enumerate}

\section{Rechte und Pflichten der Mitglieder}\label{rechteundpflichten}
\begin{enumerate}
    \item Alle Mitglieder haben das Recht, an sämtlichen Veranstaltungen des Rotaract Clubs teilzunehmen sowie die Rotaract Clubnadel zu führen. Das Recht zur Teilnahme an der Vollversammlung steht nur den ordentlichen und den beurlaubten Mitgliedern zu.
    \item Alle ordentlichen sowie die beurlaubten Mitglieder haben das Recht, Einsicht in die Finanzunterlagen und in die Protokolle der Clubzusammenkünfte und Vorstandssitzungen zu nehmen.
    \item Alle ehemaligen Mitglieder, die aus dem Club freiwillig ausgetreten sind oder deren Mitgliedschaft aufgrund des Erreichens der Altersgrenze geendet hat, haben das Recht, an sämtlichen Veranstaltungen des Clubs teilzunehmen.
    \item Jedes Mitglied ist verpflichtet, sich aktiv für den Club und seine Veranstaltungen und Projekte einzusetzen. Jedes ordentliche Mitglied hat wenigstens einmal im Jahr wesentlich an der Gestaltung einer Veranstaltung oder an der Organisation eines Projektes des Clubs mitzuwirken und wenigstens einmal im Semester an einem Sozialprojekt des Clubs teilzunehmen.
    \item Alle Mitglieder sind verpflichtet, die Interessen des Clubs nach Kräften zu fördern und alles zu unterlassen, wodurch das Ansehen des Clubs beeinträchtigt werden könnte.
    \item Die Statuten in der jeweils gültigen Fassung sowie Beschlüsse des Vorstands und der Vollversammlung sind für alle Mitglieder verbindlich.
    \item Die ordentlichen Mitglieder sind verpflichtet, im Clubjahr mindestens 60 \% aller ordentlichen Zusammenkünfte des Clubs zu besuchen. Die Teilnahme an einer ordentlichen Zusammenkunft wird nur dann anerkannt, wenn das Aktivmitglied spätestens 15 Minuten nach dem angesetzten Beginn erscheint und mindestens eine Stunde lang teilnimmt. Für die Berechnung der 60 \% wird auch der Besuch von Veranstaltungen anderer Rotaract und Rotary Clubs herangezogen, wenn das ordentliche Mitglied im Clubjahr wenigstens an 30 \% aller ordentlichen Zusammenkünfte des Rotaract Clubs Wien-Belvedere teilgenommen hat. Der Besuch einer Veranstaltung eines anderen Rotaract Clubs oder eines Rotary Clubs ist dem Sekretär durch schriftliche Bestätigung dieses Clubs nachzuweisen. Kommt ein ordentliches Mitglied regelmäßig seiner \textbf{Präsenzverpflichtung} nicht nach, so ist dieser Umstand dem betreffenden Mitglied durch den Vorstand mitzuteilen. Wird die Verpflichtung trotz Mitteilung im laufenden Clubjahr dennoch nicht erbracht, kann der Vorstand bei der nächsten Vollversammlung den Antrag auf Ausschluss des betreffenden Mitgliedes einbringen.
    \item Ist ein ordentliches Mitglied verhindert, an einer Zusammenkunft oder Veranstaltung des Clubs teilzunehmen, hat er dies dem Sekretär zeitgerecht vor dem festgesetzten Termin bekannt zu geben.
    \item Jedes Mitglied hat dem Sekretär seine vollständigen Kontaktdaten (Name, Adresse, Geburtsdatum, berufliche Tätigkeit, Telefonnummer und Emailadresse) bekannt zu geben und ihm Änderungen derselben unverzüglich anzuzeigen.
    \item Jedes Mitglied ist verpflichtet, dem Vorstand seine Mitgliedschaft in einer anderen Organisation oder Verein zu melden, deren Ziele mit der Überparteilichkeit, Unabhängigkeit oder Überkonfessionalität des Clubs nicht vereinbar sind.
    \item Bei Beendigung der Mitgliedschaft ist die Clubnadel, der Clubwimpel und alle sonstigen Gegenstände und Dokumente, die dem Rotaract Club Wien-Belvedere gehören, an den Vorstand zurückzugeben.
\end{enumerate}

\section{Mitgliedsbeiträge}
\begin{enumerate}
    \item Der Mitgliedsbeitrag setzt sich zusammen aus dem Clubbeitrag, dem Beitrag an Rotary International, dem Beitrag an den Distrikt 1910 und einer Spende an die Rotary-Foundation.
    \item Für unterstützende Mitglieder besteht der Mitgliedsbeitrag ausschließlich aus dem Clubbeitrag.
    \item Die Höhe des von den Mitgliedern jährlich zu entrichtenden Mitgliedsbeitrags wird von der Mitgliederversammlung beschlossen. Der Beitrag an Rotary International, der Beitrag an den Distrikt 1910 und die Spende an die Rotary-Foundation sind bei der Festsetzung der Höhe des Mitgliedsbeitrages zu berücksichtigen.
    \item Die Höhe des Clubbeitrages kann vom Vorstand für ein Mitglied maximal auf die Hälfte des für die Mitglieder bestimmten Betrages herabgesetzt oder für einen bestimmten Zeitraum (auch teilweise) gestundet werden, sofern das betreffende Mitglied dies schriftlich beantragt. In diesem Antrag müssen die Gründe für die Herabsetzung beziehungsweise Stundung ausführlich dargelegt werden. Sieht der Vorstand die Gründe für die Herabsetzung beziehungsweise Stundung des Clubbeitrages als gerechtfertigt an, ist dem Antrag zu entsprechen. Die Beurlaubung eines Mitglieds gilt nicht als Grund für eine Herabsetzung des Beitrags.
    \item Der Vorstand kann bestimmen, dass ordentliche Mitglieder in jenem Clubjahr, in dem sie aufgenommen wurden, von der Verpflichtung zur Leistung des Clubbeitrages zur Gänze oder teilweise befreit werden. Erfolgt die Aufnahme nach dem 30.04. des Clubjahres, findet diese Regelung auch für das darauffolgende Clubjahr Anwendung.
    \item Jedes ordentliche Mitglied sowie jedes beurlaubte Mitglied hat den festgesetzten Mitgliedsbeitrag für das Clubjahr nach Aufforderung durch den Schatzmeister innerhalb der von diesem festgesetzten Frist, jedoch auch ohne Aufforderung spätestens bis zum 1. November zu entrichten. Wird der Beitrag nicht fristgerecht bezahlt, so ist das Mitglied vom Schatzmeister unter Setzung einer Nachfrist von zwei Wochen zu mahnen. Ab der zweiten Mahnung ist eine Mahngebühr in Höhe von mindestens EUR 5,- einzuheben. Bei Beendigung der Mitgliedschaft sind noch ausstehende Verbindlichkeiten gegenüber dem Club zu tilgen.
    \item Der Vorstand kann bestimmen, dass beurlaubte Mitglieder in jenem Clubjahr, in dem sie beurlaubt wurden, von der Verpflichtung zur Leistung des Clubbeitrages zur Gänze oder teilweise befreit werden.
\end{enumerate}

\section{Beendigung der Mitgliedschaft}
\begin{enumerate}
    \item Die ordentliche Mitgliedschaft endet durch freiwilligen Austritt, durch Ausschluss aus dem Club, durch Streichung aus dem Mitgliederverzeichnis oder durch Überführung in eine unterstützende Mitgliedschaft.
    \item\label{ausschluss} Der Ausschluss eines ordentlichen oder außerordentlichen Mitgliedes kann lediglich aufgrund der Nichtbeachtung dieser Satzung, insbesondere der Nichterfüllung der in \ref{rechteundpflichten} aufgezählten Pflichten, erfolgen. Liegt nach Ansicht eines Mitgliedes ein Ausschlusstatbestand vor, so hat dieses Mitglied einen Antrag auf Ausschluss an die Vollversammlung zu stellen. Das betreffende Mitglied gilt als ausgeschlossen, wenn bei einer Vollversammlung zwei Drittel der stimmberechtigten Mitglieder dem Ausschlussantrag zustimmen. Die Beschlussfähigkeit dieser Vollversammlung ist in \ref{mehrheiten} und \ref{vertagung} festgelegt. 
    \item Der Vorstand kann bei besonders schweren Pflichtverstößen den Ausschluss eines Mitgliedes in rotarischen Medien veröffentlichen. 
    \item Die Streichung eines Mitglieds aus dem Mitgliederverzeichnis durch den Vorstand erfolgt, wenn das Mitglied über einen Zeitraum von mindestens einem Jahr weder den festgesetzten Mitgliedsbeitrag pünktlich entrichtet, noch an Zusammenkünften oder Veranstaltungen des Rotaract Clubs Wien-Belvedere teilgenommen hat und auch keine Beurlaubung oder sonstige Entschuldigung vorliegt. Der Vorstand hat, sofern eine Möglichkeit zur Kontaktaufnahme besteht, dem betreffenden Mitglied die Streichung aus dem Mitgliederverzeichnis wenigstens einen Monat vorher anzukündigen und Gelegenheit zur Stellungnahme zu geben. Auf Antrag des betreffenden Mitglieds ist die Streichung aus dem Mitgliederverzeichnis vorläufig nicht durchzuführen und der nächsten Vollversammlung gemäß \ref{ausschluss} zur endgültigen Entscheidung vorzulegen.
\end{enumerate}

\section{Clubjahr}
Das Clubjahr beginnt mit 1. Juli und endet am 30. Juni des folgenden Kalenderjahres.

\section{Veranstaltungen des Clubs}
\begin{enumerate}
    \item Mit Ausnahme der ersten drei Monate des Clubjahres und im Februar finden ordentliche Zusammenkünfte des Clubs mindestens zweimal monatlich statt. Daneben kann der Club auch außerordentliche Zusammenkünfte abhalten, bei denen es keine Anwesenheitspflicht gibt. Der Vorstand muss außerordentliche Zusammenkünfte ausdrücklich als solche bestimmen, im Zweifel sind alle Veranstaltungen ordentliche Zusammenkünfte mit Anwesenheitspflicht.
    \item Zeit und Ort der Veranstaltungen werden vom Vorstand festgelegt und sollen möglichst allen Mitgliedern gelegen kommen. 
\end{enumerate}

\section{Organe des Clubs}
Organe des Clubs sind die Vollversammlung, der Vorstand, zwei Rechnungsprüfer und das, nur im Bedarfsfall gebildete, Schiedsgericht.

\section{Die Vollversammlung}
\begin{enumerate}
    \item Die Vollversammlung gilt als Mitgliederversammlung im Sinne des Vereinsgesetzes 2002. Vollversammlungen finden nach Bedarf, jedoch mindestens zwei Mal pro Clubjahr statt. Die Einberufung einer Vollversammlung erfolgt schriftlich unter Bekanntgabe der Tagesordnung mindestens vierzehn Tage vor dem Termin durch den Präsidenten. Der Präsident ist zur Einberufung der Vollversammlung verpflichtet, wenn mindestens ein Zehntel der ordentlichen Mitglieder oder die Rechnungsprüfer dies beantragen. Der Präsident kann verlangen, dass dieser Antrag schriftlich gestellt und begründet wird.
    \item Das Recht zur Teilnahme an der Vollversammlung, das Stimmrecht sowie das aktive Wahlrecht stehen nur den anwesenden ordentlichen sowie den anwesenden beurlaubten Mitgliedern zu. Jedes stimmberechtigte Mitglied hat das Recht, Anträge in der Vollversammlung zu stellen.
    \item In die Zuständigkeit der Vollversammlung fallen alle ihr in dieser Satzung ausdrücklich zugewiesenen Angelegenheiten, insbesondere:
    \begin{enumerate}
        \item die Wahl und Enthebung des Vorstandes und die Bestellung der Rechnungsprüfer;
        \item die Entscheidung über Aufnahme von ordentlichen Mitgliedern nach \ref{abgelehnt} und über den Ausschluss von Mitgliedern;
        \item Änderungen und Ergänzungen der Statuten;
        \item die Auflösung des Clubs;
        \item die Entgegennahme und Genehmigung des Jahresberichts des Vorstands und der Jahresabrechnung sowie die Entlastung des Vorstands;
        \item die Festsetzung des Mitgliedsbeitrages für ordentliche und beurlaubte Mitglieder;
        \item die Verleihung und Aberkennung der Ehrenmitgliedschaft.
    \end{enumerate}
    \item\label{mehrheiten} Die Entscheidungen der Vollversammlung haben, soweit in diesen Statuten nichts anderes bestimmt ist, mit einfacher Mehrheit der anwesenden stimmberechtigten Mitglieder zu erfolgen. Die Vollversammlung ist bei Anwesenheit von mindestens der Hälfte der ordentlichen Mitglieder voll beschlussfähig. Ist die Vollversammlung zur festgesetzten Zeit nicht beschlussfähig, so findet sie 15 Minuten später mit derselben Tagesordnung statt. In diesem Fall ist die Beschlussfähigkeit ohne Rücksicht auf die Zahl der erschienenen Mitglieder gegeben, doch darf über folgende Anträge nicht abgestimmt werden:
    \begin{enumerate}
        \item Ausschluss von Mitgliedern gemäß \ref{ausschluss};
        \item Statutenänderungen gemäß \ref{statutenaenderung};
        \item Auflösung des Clubs gemäß \ref{aufloesung}.
    \end{enumerate}
    Solche Anträge müssen aber in jedem Falle verlesen und bei der nächsten Vollversammlung neuerlich vorgelegt werden.
    \item\label{vertagung} Kann der Ausschluss von Mitgliedern, eine Statutenänderung oder die Auflösung des Clubs bei der ersten hierfür anberaumten Vollversammlung mangels Beschlussfähigkeit nicht beschlossen werden, so kann dafür mindestens 14 Tage später eine neuerliche Vollversammlung angesetzt werden. Diese ist zum Ausschluss von Mitgliedern, zu Statutenänderungen und zur Auflösung des Clubs auch dann beschlussfähig, wenn nach einer Vertagung um 15 Minuten nicht mindestens die Hälfte der ordentlichen Mitglieder anwesend sind. 
\end{enumerate}

\section{Der Vorstand}
\begin{enumerate}
    \item Der Vorstand ist das Leitungsorgan im Sinne des Vereinsgesetzes 2002 und besteht aus dem Präsidenten, dem Pastpräsidenten (Präsident des unmittelbar vorangegangenen Clubjahres), dem Vizepräsidenten, dem Sekretär, dem Clubmeister, dem Schatzmeister, dem Sozialreferenten und dem Auslandsreferenten.
    \item Die Funktionsdauer des Vorstandes beträgt ein Clubjahr. Die Vorstandssitzungen finden zumindest einmal im Monat oder bei Bedarf auch öfter statt und werden vom Präsidenten einberufen. Sie sind allen ordentlichen und beurlaubten Mitgliedern zugänglich.
    \item Der Vorstand ist bei Anwesenheit von mindestens vier Vorstandsmitgliedern, darunter der Präsident, beschlussfähig. Der Vorstand fasst seine Beschlüsse mit einfacher Mehrheit. Bei Stimmengleichheit entscheidet die Stimme des Präsidenten.
    \item In dringenden Angelegenheiten kann der Vorstand auch Beschlüsse fassen, ohne zu einer Vorstandssitzung zusammen zu kommen. Diese Beschlüsse bedürfen der Zustimmung von mindestens fünf Vorstandsmitgliedern, darunter dem Präsidenten. Nach Möglichkeit sind alle Vorstandsmitglieder zu kontaktieren. Liegt nicht die Zustimmung aller Mitglieder des Vorstands vor, sind die Beschlüsse ausführlich zu dokumentieren und bei der nächsten Vorstandssitzung vorzutragen. Der Präsident muss dem Vorstand die Gründe für die eilige Beschlussfassung darlegen und alle übrigen anwesenden Vorstandsmitglieder müssen diese Gründe als gerechtfertigt ansehen, andernfalls gelten diese Beschlüsse als nicht gültig zustande gekommen.
    \item Die Vollversammlung kann jederzeit den gesamten Vorstand oder einzelne seiner Mitglieder ihrer Funktion entheben. Der Rücktritt des Vorstandes oder eines seiner Mitglieder wird erst mit der Wahl eines Nachfolgers gültig, doch ist das betroffene Vorstandsmitglied ab diesem Zeitpunkt nicht mehr vertretungsbefugt.
    \item In die Zuständigkeit des Vorstandes fallen alle Angelegenheiten, die in dieser Satzung nicht ausdrücklich der Vollversammlung zugewiesen sind, insbesondere:
    \begin{enumerate}
        \item die Verwaltung des Vermögens;
        \item die Durchführung der Beschlüsse der Vollversammlung;
        \item die Ablegung eines Rechenschaftsberichtes gegenüber der Vollversammlung;
        \item die Einsetzung von Ausschüssen.
    \end{enumerate}
    \item Der Vorstand hat in einer im letzten Monat des Clubjahres oder in einer in den ersten vier Monaten des darauffolgenden Clubjahrs abzuhaltenden Vollversammlung einen Jahresbericht über seine Tätigkeit im Clubjahr und die finanzielle Gebarung zur Annahme vorzulegen. Ferner hat der Schatzmeister eine Jahresabrechnung des Clubjahres bestehend aus einer Einnahmen- und Ausgabenrechnung samt Vermögensübersicht vorzulegen, die von den Rechnungsprüfern geprüft und für in Ordnung befunden sein muss, bevor die Vollversammlung dem Vorstand die Entlastung erteilen kann. 
\end{enumerate}

\section{Aufgaben der Vorstandsmitglieder}
\begin{enumerate}
    \item Der \textbf{Präsident} vertritt den Club nach außen. Alle Ausfertigungen und Bekanntmachungen des Clubs werden vom Präsidenten zusammen mit einem der anderen Vorstandsmitglieder rechtsverbindlich gezeichnet. Rechtsgeschäftliche Bevollmächtigungen, für den Club zu zeichnen, können vom Präsidenten zusammen mit einem anderen Vorstandsmitglied erteilt werden. Der Vizepräsident vertritt den Präsidenten bei dessen Verhinderung.
    \item Der Präsident führt bei allen ordentlichen Zusammenkünften des Clubs und des Clubvorstands den Vorsitz. Er beruft mit Genehmigung des Vorstands alle Ausschüsse und benennt bei freiwerdenden Positionen im Vorstand nach Billigung durch selbigen Interimsmitglieder bis zur nächsten ordentlichen Vorstandswahl. Weiters hat er alle erforderlichen Meldungen an die zuständige Vereinsbehörde zu erstatten.
    \item Der \textbf{Vizepräsident} übernimmt die Amtsgeschäfte des Präsidenten für den Fall, dass dieser aus welchen Gründen auch immer aus dem Amt ausscheidet, und führt in Abwesenheit des Präsidenten stellvertretend den Vorsitz bei Clubveranstaltungen, Vorstandssitzungen und Vollversammlungen.
    \item Der \textbf{Clubsekretär} führt das Clubarchiv und fungiert bei Zusammenkünften und Sitzungen von Club und Clubvorstand als Protokollführer.
    \item Der \textbf{Clubmeister} gestaltet die Veranstaltungen des Clubs.
    \item Der \textbf{Schatzmeister} ist für sämtliche finanzielle Angelegenheiten des Clubs zuständig. Er hat die Mitgliedsbeiträge für den Club einzubehalten und die dafür vorgesehenen Beiträge an den Rotary-Distrikt 1910 weiterzuleiten. Die Spenden an die Foundation sind an die Foundation abzuführen. Der Schatzmeister hat über alle Einnahmen und Ausgaben Aufzeichnungen zu führen, er erstellt die Jahresabrechnung bestehend aus einer Einnahmen- und Ausgabenrechnung samt Vermögensübersicht und übernimmt den Bankverkehr mit einer durch den Clubvorstand genehmigten Bank. Er erstattet dem Vorstand regelmäßig Bericht über die Finanzen des Clubs und gewährt auf Verlangen jedem Mitglied Einblick in die Finanzunterlagen.
    \item Der \textbf{Sozialreferent} koordiniert die Sozialprojekte des Clubs.
    \item Der \textbf{Auslandsreferent} ist für die Aufnahme und ständige Pflege der Kontakte zu ausländischen Rotaract und Rotary Clubs zuständig.
    \item Die nähere Ausgestaltung der Kompetenzverteilung zwischen den Vorstandsmitgliedern regelt der Vorstand durch Beschluss mit einfacher Mehrheit.
\end{enumerate}

\section{Bestellung des Vorstands und der Rechnungsprüfer}
\begin{enumerate}[ref=Absatz \arabic*]
    \item Die Bestellung des Vorstandes (mit Ausnahme des Präsidenten und des Pastpäsidenten) sowie der beiden Rechnungsprüfer für das folgende Clubjahr erfolgt in einer Vollversammlung des laufenden Clubjahres in geheimer, persönlicher und direkter Wahl bis spätestens 30. Juni des laufenden Clubjahres. Auch die Wahl des Präsidenten Elect für das jeweils kommende Clubjahr erfolgt bis spätestens 30. Juni. Falls die Vollversammlung dies einstimmig beschließt, können der Vorstand und die Rechnungsprüfer auch offen gewählt werden.
    \item\label{incomingpres} In das Amt des Präsidenten des folgenden Clubjahres rückt der Vizepräsident des laufenden Clubjahres automatisch nach. Der Präsident des laufenden Clubjahres bleibt als Pastpräsident Vorstandsmitglied während des folgenden Clubjahres. Während der Amtsdauer ausscheidende Mitglieder des Vorstands werden mittels Nachwahl von der Vollversammlung ersetzt.
    \item Der Präsident ist nur dann zu wählen, wenn die Automatik des \ref{incomingpres} -- aus welchen Gründen auch immer -- nicht anwendbar ist. Wenn der Clubpräsidenten während seiner Amtszeit das 34. Lebensjahr erreicht, darf er ein weiteres Jahr als Pastpräsident zur Verfügung stehen, um eine kontinuierliche Amtsführung zu gewährleisten.
    \item Jedes ordentliche Mitglied sowie jedes beurlaubte Mitglied, dessen Beurlaubung zum Zeitpunkt des Amtsantritts beendet sein wird, kann zum Mitglied des Vorstandes und zum Rechnungsprüfer gewählt werden. Die Rechnungsprüfer dürfen keinem Organ mit Ausnahme der Vollversammlung angehören, dessen Tätigkeit Gegenstand der Prüfung ist. Sofern der Club nicht weniger als 15 ordentliche Mitglieder hat, darf außer dem Pastpräsidenten kein Mitglied des Vorstands mehr als eine Funktion gleichzeitig ausüben.
    \item Jedes Mitglied, darf abgesehen von der Funktionsdauer als Präsident, Vizepräsident und Pastpräsident höchstens drei Jahre in Folge Mitglied des Vorstands sein. Präsident darf jedes Mitglied nur höchstens ein Jahr sein. Diese Einschränkungen gelten jedoch nicht, wenn der Club weniger als 25 ordentliche Mitglieder hat.
    \item Wahlvorschläge für die Mitglieder des Vorstands und die Rechnungsprüfer können in Absprache mit den vorgeschlagenen Kandidaten von jedem stimmberechtigten Mitglied bis spätestens eine Woche vor der Wahl an den Vorstand eingereicht werden. Der Präsident des folgenden Clubjahres hat Wahlvorschläge für sämtliche Mitglieder des Vorstands und die Rechnungsprüfer einzubringen. Bekommt kein Wahlvorschlag in der Vollversammlung die erforderliche Mehrheit, so können auch weitere Wahlvorschläge direkt in der Vollversammlung eingebracht werden.
    \item Ein Wahlvorschlag gilt als angenommen, wenn ihm mindestens die Hälfte der anwesenden stimmberechtigten Mitglieder zustimmt. Eine Stimmenthaltung gilt nicht als Zustimmung. Erhält keiner der Kandidaten die Zustimmung mindestens der Hälfte der anwesenden stimmberechtigten Mitglieder, so haben sich die beiden Kandidaten mit der höchsten Stimmenanzahl einer Stichwahl zu unterziehen, bei der eine einfache Stimmenmehrheit ausreicht.
    \item Die beiden Rechnungsprüfer werden von den stimmberechtigten Mitgliedern im Rahmen einer Vollversammlung für die Dauer eines Clubjahres bestellt. Die Bestellung erfolgt einzeln mit einfacher Mehrheit.
\end{enumerate}

\section{Aufgaben der Rechnungsprüfer}
\begin{enumerate}
    \item Die zwei gewählten Rechnungsprüfer sind für die Kontrolle der finanziellen Angelegenheiten des Clubs verantwortlich. Sie haben die Finanzgebarung des Vereins im Hinblick auf die Ordnungsmäßigkeit der Rechnungslegung und die statutengemäße Verwendung der Mittel zu prüfen. Der Vorstand ist verpflichtet, den Rechnungsprüfern die erforderlichen Unterlagen vorzulegen und die erforderlichen Auskünfte zu erteilen.
    \item Der Prüfungsbericht hat die Ordnungsmäßigkeit der Rechnungslegung und die statutengemäße Verwendung der Mittel zu bestätigen oder festgestellte Gebarungsmängel oder Gefahren für den Bestand des Vereins aufzuzeigen. Auf ungewöhnliche Einnahmen oder Ausgaben, vor allem auf Insichgeschäfte, ist besonders einzugehen.
    \item Die Rechnungsprüfer haben der Vollversammlung zu berichten. Die zuständigen Vorstandsmitglieder haben die von den Rechnungsprüfern aufgezeigten Gebarungsmängel zu beseitigen und Maßnahmen gegen aufgezeigte Gefahren zu treffen.
\end{enumerate}

\section{Das Schiedsgericht}
\begin{enumerate}
    \item Bei Streitigkeiten aus dem Vereinsverhältnis entscheidet zuerst ein Schiedsgericht, das aus fünf Personen besteht. Jede der streitenden Parteien wählt zwei ordentliche Mitglieder, diese vier wählen ein fünftes. Alle Streitparteien sind anzuhören. Die Beschlüsse des Schiedsgerichtes sind mit einfacher Mehrheit bei Anwesenheit aller Schiedsrichter zu fassen.
    \item Nach Entscheidung des Schiedsgerichts, spätestens jedoch sechs Monate nach Anrufung desselben, steht den Streitparteien der ordentliche Rechtsweg offen.
    \item Beim Schiedsgericht im Sinne dieser Statuten handelt es sich lediglich um eine Schlichtungseinrichtung im Sinne des Vereinsgesetzes 2002 und nicht um ein Schiedsgericht im Sinne der §§ 577 ff ZPO.
\end{enumerate}

\section{Änderungen der Statuten}\label{statutenaenderung}
\begin{enumerate}
    \item Eine Änderung dieser Statuten kann nur durch die Vollversammlung mit einfacher Mehrheit der anwesenden stimmberechtigten Mitglieder erfolgen. Die Beschlussfähigkeit dieser Vollversammlung ist in \ref{mehrheiten} und \ref{vertagung} festgelegt.
    \item Jede Änderung der Statuten durch Beschluss der Vollversammlung ist dem Rotary Patenclub zur Kenntnis zu bringen, der dagegen innerhalb eines Monats Einspruch erheben kann.
\end{enumerate}

\section{Auflösung des Vereins}\label{aufloesung}
\begin{enumerate}
    \item Die freiwillige Auflösung des Clubs erfolgt durch eine hierfür anzuberaumende Vollversammlung. Die Beschlussfähigkeit dieser Vollversammlung ist in \ref{mehrheiten} und \ref{vertagung} festgelegt. Der Antrag auf Auflösung des Clubs gilt als angenommen, wenn in einer dafür beschlussfähigen Vollversammlung zwei Drittel der stimmberechtigten Mitglieder dem Antrag zustimmen.
    \item Der Club ist aufzulösen, wenn Rotary International die Anerkennung als Rotaract Club entzieht. Im Fall der Auflösung des Rotary Patenclubs oder des Entzuges der Patenschaft durch den Patenclub kann ein neuer Rotary Club die Patenschaft übernehmen. Falls sich jedoch innerhalb eines Jahres kein solcher Nachfolger finden lässt, muss der Rotaract Club aufgelöst werden. Die Liquidation findet durch den Vorstand statt.
    \item Das bei der Auflösung vorhandene aktive Clubvermögen fließt einer durch die Vollversammlung zu bestimmenden wohltätigen Organisation zu.
\end{enumerate}

\section{Übergangs- und Schlussbestimmungen}
\begin{enumerate}
    \item Die vorliegenden Statuten ersetzen nach Inkrafttreten die bisherigen Statuten.
    \item \glqq{}Schriftlich\grqq{} im Sinne dieser Satzung schließt ausdrücklich die Übertragung des Schriftsatzes in jeder technisch möglichen Form ein, insbesondere auf dem Wege der automationsunterstützten Datenübertragung (Email) oder per Telefax.
    \item Soweit personenbezogene Bezeichnungen nur in männlicher Form angeführt sind, beziehen sie sich auf Frauen und Männer in gleicher Weise.
\end{enumerate}

\end{document}

% https://tex.stackexchange.com/a/122936/318292

% TODO Änderungen: rm Telefax?, evtl. Organs as subsections? chk Duplication, chk Altergrenze auch indirekt erwähnt nach Elevation. Mitgliedsbeiträge usw, chk auch Erwähnungen von Festlegen bei GV usw, chk auch unterstützende etc.; ref abgelehnt sollte lit c ??
